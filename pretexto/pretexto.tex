
% ---
% Capa
% ---
\imprimircapa
% ---

% ---
% Folha de rosto
% (o * indica que haverá a ficha bibliográfica)
% ---
\imprimirfolhaderosto*
% ---

% ---
% Inserir a ficha bibliografica
% ---

% Isto é um exemplo de Ficha Catalográfica, ou ``Dados internacionais de
% catalogação-na-publicação''. Você pode utilizar este modelo como referência. 
% Porém, provavelmente a biblioteca da sua universidade lhe fornecerá um PDF
% com a ficha catalográfica definitiva após a defesa do trabalho. Quando estiver
% com o documento, salve-o como PDF no diretório do seu projeto e substitua todo
% o conteúdo de implementação deste arquivo pelo comando abaixo:
%
% \begin{fichacatalografica}
%     \includepdf{fig_ficha_catalografica.pdf}
% \end{fichacatalografica}


\begin{fichacatalografica}
	\sffamily
	\vspace*{\fill}					% Posição vertical
	\begin{center}					% Minipage Centralizado
	\fbox{\begin{minipage}[c][8cm]{13.5cm}		% Largura
	\small
	\imprimirautor
	%Sobrenome, Nome do autor
	
	\hspace{0.5cm} \imprimirtitulo  / \imprimirautor. --
	\imprimirlocal, \imprimirdata-
	
	\hspace{0.5cm} \pageref{LastPage} p. : il. (algumas color.) ; 30 cm.\\
	
	\hspace{0.5cm} \imprimirorientadorRotulo~\imprimirorientador\\
	
	\hspace{0.5cm}
	\parbox[t]{\textwidth}{\imprimirtipotrabalho~--~\imprimirinstituicao,
	\imprimirdata.}\\
	
	\hspace{0.5cm}
		1. WORD1.
		2. WORD2.
		3. WORD3.
		4. WORD4.
		I. \imprimirorientador.
		II. \instituicao.
		III. \imprimirtitulo 			
	\end{minipage}}
	\end{center}
\end{fichacatalografica}
% ---


% ---
% Inserir folha de aprovação
% ---

% Isto é um exemplo de Folha de aprovação, elemento obrigatório da NBR
% 14724/2011 (seção 4.2.1.3). Você pode utilizar este modelo até a aprovação
% do trabalho. Após isso, substitua todo o conteúdo deste arquivo por uma
% imagem da página assinada pela banca com o comando abaixo:
%
% \includepdf{folhadeaprovacao_final.pdf}
%

\begin{folhadeaprovacao}


\begin{center}
	\textbf{\imprimirautor}
	
	\vspace{2\baselineskip}
	
	\textbf{\imprimirtitulo}
	
	\vspace{2\baselineskip}	
\end{center}

%\if@openright\cleardoublepage\else\clearpage\fi
%\thispagestyle{empty}
%\noindent
%Dissertação de Mestrado apresentada por \textbf{\imprimirautor} ao programa de Pós-Graduação em
%Ciência da Computação do Centro de Informática da Universidade Federal de
%Pernambuco, sob o título \textbf{\imprimirtitulo}, orientada pelo \textbf{Prof. \imprimirorientador}:

{%
	\hspace{.38\textwidth}
	\begin{minipage}{.48\textwidth}
		\imprimirpreambulo
	\end{minipage}%
	%\vspace*{\fill}
}%

\vspace{2\baselineskip}	

\noindent
Aprovado em: 00/00/2017.
%\vspace{3\baselineskip}

\vspace{2\baselineskip}	

\begin{center}

{\bfseries{BANCA EXAMINADORA}}
\vspace{3\baselineskip}

-----------------------------------------------------------------------\\
NOME DO PROFESSOR 01\\
Centro de Informática/UFPE

\vspace{\baselineskip}
\vspace{\baselineskip}
\vspace{\baselineskip}
-----------------------------------------------------------------------\\
NOME DO PROFESSOR 02\\
Centro de Informática/UFPE


\vspace{\baselineskip}
\vspace{\baselineskip}
\vspace{\baselineskip}
-----------------------------------------------------------------------\\
NOME DO PROFESSOR 03\\
Centro de Informática/UFPE\\
(\bfseries{Orientador})

\vfill
%\large\imprimirlocal
%\\
%\large\imprimirdata

\end{center}

\end{folhadeaprovacao}

% ---

% ---
% Dedicatória
% ---
%\begin{dedicatoria}
%   \vspace*{\fill}
%   \centering
%   \noindent
%   \textit{ Este trabalho é dedicado às crianças adultas que,\\
%   quando pequenas, sonharam em se tornar cientistas.} \vspace*{\fill}
%\end{dedicatoria}
% ---

% ---
% Agradecimentos
% ---

\begin{agradecimentos}

\lipsum[1-2]

\end{agradecimentos}

% ---

% ---
% Epígrafe
% ---
%\begin{epigrafe}
%    \vspace*{\fill}
%	\begin{flushright}
%		\textit{``Não vos amoldeis às estruturas deste mundo, \\
%		mas transformai-vos pela renovação da mente, \\
%		a fim de distinguir qual é a vontade de Deus: \\
%		o que é bom, o que Lhe é agradável, o que é perfeito.\\
%		(Bíblia Sagrada, Romanos 12, 2)}
%	\end{flushright}
%\end{epigrafe}
% ---


% ---
% RESUMOS
% ---

% resumo em português
\setlength{\absparsep}{18pt} % ajusta o espaçamento dos parágrafos do resumo


\begin{resumo}

	\lipsum[2-2]
	
	\textbf{Palavras-chave}: WORD1. WORD2. WORD3. WORD4.
\end{resumo}
% resumo em inglês
\begin{resumo}[Abstract]
	\begin{otherlanguage*}{english}
		\lipsum[2-2]
		
		\vspace{\onelineskip}
		\noindent 
		\textbf{Keywords}: WORD1. WORD2. WORD3. WORD4.
	\end{otherlanguage*}
\end{resumo}